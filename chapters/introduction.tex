\chapter{Introduction}
\label{sec:intro}
Emerging situation-awareness applications such as public safety via large-scale video-analytics and connected/autonomous driving applications process sensor data about the environment from multiple sensors and aim to respond in real-time to events. Such applications rely on platform services such as application orchestrators, publish-subscribe systems, and key-value stores. Although offerings of such platform services are quite popular in the datacenter space, e.g., Kubernetes, Apache Pulsar and Apache Cassandra, they have not been used over a continuum of geo-distributed edge resources. Edge platforms such as Azure IoT Edge and AWS Greengrass focus on providing such services on each edge site, which is independent from other edge sites, and any interaction across edge sites goes through a datacenter.
\section{Problem Statement}
This thesis aims to ensure the seamless and efficient operation of platform services on a geo-distributed edge infrastructure so that situation-awareness applications can benefit from proximal nature of edge infrastructure and meet application's performance requirements. 
\par Providing platform services that span multiple geo-distributed edge sites is challenging due to the inherent dynamism in  situation-awareness applications and thus requires continuous adaptation. (1) Inter-client coordination (e.g., in collaborative autonomous driving) and access to data is based on spatial proximity. Each client's mobility necessitates continuously adapting the set of other clients that it coordinates with and the data it accesses. (2) Edge infrastructure is highly heterogeneous both in terms of network topology (inter-edge-site and client-edge-site network latency) and per-edge-site resource capacity, making topology and resource-awareness crucial to control-plane policies. Continuous client mobility affects experienced quality of service (QoS) due to changes in routing to get to the desired edge-site. (3) Performance degradation can be experienced by an application due to various reasons, such as inflation of last-mile latency due to client's movement, decrease in available uplink bandwidth in edge-cloud connection, increased processing workload, etc. Fine-grained end-to-end monitoring of application instances is required to detect the source of performance degradation and trigger the right adaptation.
\par Dealing with the aforementioned challenges requires the control-plane of platform services (e.g., broker selection policy for topic-based publish-subscribe systems) to be aware of the dynamism of clients and infrastructure. This thesis proposes three mechanisms that provide relevant information to the control plane of a platform service, such that the control-plane can take actions and continuously satisfy client's performance requirements.
\begin{itemize}
\item Mechanism for specifying the spatial context of system entities (compute/data), which guides their placement over the infrastructure and clients' data access. A spatial KD-Tree data structure is used to compute Region-of-Interest and map clients to compute/data. 
\item Mapping geo-location to network proximity space in order to perform topology-aware compute/data placement. A decentralized peer-to-peer network coordinate protocols is used to estimate network proximity.
\item End-to-end monitoring of application instances for detecting the specific performance bottleneck and triggering the right reconfiguration action. The proposed monitoring subsystem incorporates application-specific metrics aggregation and alert generation policies. 
\end{itemize}

\section{Thesis Statement}
The aforementioned mechanisms are fundamental building blocks for the control-plane of platform services that operate on a geo-distributed infrastructure. Furthermore, the implementation of these mechanisms can be done such that they are efficient and scalable in a geo-distributed setting.
\section{Contributions}
In this thesis I present the architecture and implementation of aforementioned mechanisms as well as how various system components interact with them. Furthermore, I illustrate the aforementioned mechanisms using control-plane policies in 3 different contexts. I demonstrate their efficacy by implementing the proposed control-plane policies in platform services and evaluating observed performance of typical situation awareness applications.
\begin{itemize}
\item An application orchestration platform, OneEdge, that performs compute placement using end-to-end latency requirements and network proximity information. In the case of applications needing inter-client coordination, developers are allowed to specify the spatial context of clients that should be served by the same application instance. Continuous monitoring of spatial context and observed performance metrics with custom policies detects SLO violations and triggers migration of application components. 
%OneEdge uses mechanism M1 to  partition clients among applications instances of a coordinated application. It assigns a spatial context to each application instance and all clients belonging to that context are served by the application instance.
%OneEdge uses mechanism M2 to map geo-distributed edge servers and clients to network proximity space. This mapping enables the latency-aware selection of an edge server to host a given application instance with a given spatial context. It also enables the estimation of network latency between any pair of nodes, which is useful information for end-to-end placement of application instances.
%OneEdge uses application-instance level monitoring data (M3) to determine observed end-to-end application processing latencies. SLO violation detection policies process the monitoring data to detect violations and trigger an appropriate reconfiguration.
%Scheduling policies in OneEdge consume a combination of historical (short-term) monitoring data to predict client mobility and spatial distribution of request arrival rates. These insights, along with current resource commitments, are used to rank/filter nodes in the scheduling policy. (M4)

\item A topic-based publish-subscribe system, ePulsar, that performs topic (data) placement among brokers on edge sites based on network proximity information to satisfy end-to-end message delivery latency requirements. Developers use the spatial context abstraction to specify whether a topic is relevant for a given client, such that topic placement is guided only by relevant clients’ locations. Topic migration is triggered when monitoring subsystem detects violation of client-broker latency.
%M1: For applications whose per-instance functionality is tied to a specific geographical area, ePulsar uses M1 to associate topics corresponding to that instance to a spatial context. 
%M2: Client-broker latencies are estimated using network coordinates, which is required for selection of broker to host a topic. 
%M3: Observed end-to-end message delivery latency per-topic needs to be monitored to detect violations. 
%M4: Prediction of spatial distribution of topics and message rates across brokers as well as trajectory forecasts of clients is needed to select the broker on which a topic should be hosted. 

\item A key-value store, FogStore, that meets a developer-specified tradeoff between latency, consistency and fault tolerance. Developers specify the spatio-temporal context of data items using which FogStore determines optimal data placement and consistency level for clients. FogStore is able to provide consistent access with low latency by exploiting the spatial-locality in data access patterns of applications. 
%M1: K-V pairs correspond to a specific application entity, which has a Region of Interest. The region of interest defines the bounds on the location of clients that need to have real-time consistent access to those K-V pairs. 
%M2: Client to datastore-node latencies are estimated using network coordinates. 
%M3: Monitoring per-KV request arrival rate history.
\end{itemize}

Through my research, I demonstrate the efficacy and scalability of these mechanisms by utilizing them in the context of three different platform services - an application orchestrator OneEdge, a publish-subscribe system ePulsar and a key-value store FogStore.

\section{Roadmap}
\Cref{sec:background} presents background and core concepts involved in this dissertation. 
\Cref{sec:mechanisms} presents the design principles and architecture of the proposed mechanisms and presents microbenchmarks to evaluate their efficacy and scalability in a geo-distributed setting. 
\Cref{sec:epulsar} describes the topic-based publish-subscribe system \epulsar and it's interaction with the mechanisms to ensure that topic placement across brokers is aware of end-to-end latency requirements of the clients. \Cref{sec:oneedge} shows how the application orchestrator OneEdge uses the proposed mechanisms to perform application placement and migration to meet application requirements. Finally, \Cref{sec:plan} shows the timeline for completing the thesis.