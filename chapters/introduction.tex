\chapter{Introduction}
\label{sec:intro}

Situation-awareness applications sense the physical environment, extract actionable information from it, and perform action based on the extracted information. They perform critical tasks such as navigation control for unmanned aerial vehicles (UAVs), autonomous vehicle control, Pan-Tilt-Zoom (PTZ) tuning for connected cameras, where response time of the application as perceived by the end-client (UAV or connected camera) is crucial for ensuring correct functionality. Hence, instances of such applications need to be deployed in close network proximity from end-clients instead of Cloud datacenters to ensure that network traversal through the wide-area network (WAN) does not adversely impact response time. Edge computing has gained prominence as a computing paradigm that utilizes computational and storage resources at the edge of the network, thereby allowing application instances to be deployed across a continuum of resources ranging from access networks to datacenters \cite{ramachandran2021case}. Utilizing Edge infrastructure for hosting situation-awareness applications would allow them to achieve predictable and low response times.
\par Managing situation-awareness applications requires deployment and scaling of application instances based on client demand, which necessitates the use of an application orchestrator such as Kubernetes. These applications possess a number of communicating entities which exchange information that is integral to its functionality. This naturally lends itself to using a publish-subscribe system, such as Apache Pulsar, to enable such a communication efficiently. These applications also need to store state, which is used to guide their future actions. They need access to a database, such as Apache Cassandra, to store and query application state. Access to these platform services is in the critical path of the application logic of our target applications. Hence, the platform service instances need to be deployed on Edge resources to avoid high communication overhead when accessing them. Situation-awareness applications require that access to platform services does not introduce significant overhead which would affect response time. Furthermore, these applications  have a strong dependence on spatial location for mapping end-devices to application instance, defining communication patterns, etc. 

\subsection{Problem Statement}

Although the data-plane of contemporary platform services offer intuitive semantics and high performance, their control-plane is designed for operation in datacenters. Edge infrastructure has a unique set of challenges that are not present in the Cloud computing space.
\begin{itemize}
    \item Firstly, the network topology of Edge infrastructure is highly heterogeneous, with high variability in network latency between Edge sites. This is unlike datacenters, where nodes are connected together by a fast and low-latency interconnect. Hence, network latency between clients and platform service nodes, and between platform service nodes is assumed to be negligible in cloud-based platform service deployments. Using such network-latency-agnostic control-plane policies in platform services deployed at the Edge would result in high overheads due to network latency in the critical path of applications.
    \item Secondly, contemporary platform services do not consider client location as a first-class factor for making control-plane decisions. Therefore, it becomes the application developer's burden to ensure that client-to-application mapping and communication between system components is done in a location-aware manner.
    \item Finally, client mobility creates significant dynamism in the input workload. The network connectivity of clients changes as a result of mobility, which frequently results in the current mapping of client to application instance unsuitable for meeting low response time requirements. A change in client location would require the client interact with a different set of clients or access state corresponding to the new spatial area it is in. Furthermore, client mobility also results in the occurrence of skews in workload distribution which could create performance hotspots certain platform service nodes. To cater to these dynamisms, the platform services need to monitor all latency overheads and make reconfiguration decisions in the case of violation of response time requirements.
\end{itemize}


\subsection{Thesis Statement}
In order to solve the challenges faced when designing control-plane policies for Edge-based platform services, this dissertation proposes three mechanisms that provide relevant information to the control plane of a platform service, such that it can take actions and continuously satisfy client's performance requirements.
\begin{itemize}
\item Mechanism for specifying the spatial affinity of system entities (compute/data), which guides their placement over the infrastructure and clients' data access. 
\item Mapping geo-location to network proximity space in order to perform network-latency-aware compute/data placement. A decentralized peer-to-peer network coordinate protocols is used to estimate network proximity.
\item End-to-end monitoring of application instances for detecting the specific performance bottleneck and triggering the right reconfiguration action. The proposed monitoring subsystem incorporates application-specific metrics aggregation and alert generation policies. 
\end{itemize}

The aforementioned mechanisms are fundamental building blocks for the control-plane of platform services that operate on a geo-distributed infrastructure. Furthermore, the implementation of these mechanisms can be done such that they are efficient and scalable in a geo-distributed setting.

\subsection{Contributions}
This thesis describes the interface offered by each of the three proposed mechanisms and how they provide crucial information for control-plane policy decision making. It presents a design space exploration of each of the mechanisms, evaluating each design choice in terms of efficacy, efficiency and scalability. The applicability of the proposed mechanisms is then demonstrated by using them to build control-plane policies for three edge-centric platform services and evaluating the observed performance of typical situation awareness applications. These platform services are described as follows.
\begin{itemize}
\item An application orchestration platform, \oneedge{}, that performs application placement using response time requirements and network proximity information. It also performs location-aware client to application instance mapping for those applications that possess spatial affinity. Continuous monitoring of client location and observed response time with custom policies detects violations of application requirements and triggers migration of client to a different application instance. 
\item A topic-based publish-subscribe system, \textbf{ePulsar}, that performs topic (data) placement among brokers on edge sites based on network proximity information to satisfy end-to-end message delivery latency requirements. Topic migration is triggered when monitoring subsystem detects violation of end-to-end message delivery latency.
\item A key-value store, \textbf{FogStore}, that meets a developer-specified tradeoff between latency, consistency and fault tolerance. Developers specify the spatio-temporal context of data items using which FogStore determines optimal data placement and consistency level for clients. FogStore is able to provide consistent access with low latency by exploiting the spatial-locality in data access patterns of applications. 
\end{itemize}

\subsection{Roadmap}
The remainder of this document is structured as follows. Chapter 2 discusses the target application space, i.e., situation-awareness applications, including their general characteristics, specific examples and the requirements they pose on the platform services. The chapter also covers how these requirements can only be fulfilled by the introduction of new mechanisms into the control-plane of platform services. Next, Chapter 3 describes these mechanisms concretely, including the abstractions that they expose to control-plane policies, and results from a set of experimental evaluations that quantify the possible improvement in control-plane decisions if these mechanisms are used. Chapter 4 presents a design-space exploration of each of the mechanisms, wherein it quantitatively compares multiple designs for each mechanism in terms of efficiency and scalability. Chapters 5, 6 and 7 demonstrate the use of the proposed mechanisms in the control-plane policies of \textbf{ePulsar}, \oneedge{} and \textbf{FogStore} respectively, as mentioned above. Chapter 8 presents the related work and their connection with this dissertation. Chapter 9 discusses the ideas and lessons learned by carrying out the research presented in this dissertation. Finally, Chapter 10 concludes the dissertation and presents directions for future research.