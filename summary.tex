\begin{summary}
Situation-awareness applications require low-latency response and impose significant network bandwidth requirements, and hence benefit from geo-distributed Edge infrastructure. These applications rely on several platform services, such as Kubernetes, Apache Cassandra and Pulsar, for managing their compute and data components. These platform services are responsible for scheduling compute components and placing data across the geo-distributed Edge infrastructure, that has strong implications on the applications’ response time. Hosting situation-awareness applications on Edge infrastructure poses novel requirements on the platform services. Firstly, the processing logic of these applications is closely tied to the physical environment that it is interacting with. Hence, the access pattern to compute and data exhibits strong spatial affinity. Secondly, the network topology of Edge infrastructure is heterogeneous, wherein communication latency forms a significant portion of the end-to-end response time of the application. Finally, clients of situation-awareness applications are inherently mobile, necessitating continuous adaptation of compute and data placement decisions to ensure latency requirements are satisfied while adhering to spatial affinity requirements.

The control planes of off-the-shelf platform services do not have the necessary primitives to incorporate spatial affinity and network topology awareness into the compute and data orchestration policies. They also do not perform fine-grained end-to-end monitoring of application response times to detect and adapt to performance degradations due to client mobility. This dissertation presents three mechanisms that inform the compute/data placement policies for platform services, so that application performance requirements can be met.
\begin{itemize}
\item Distributed spatial context management for system entities - including clients and data/compute components to ensure spatial affinity constraints are satisfied.
\item Topology awareness through scalable network proximity estimation among clients and Edge sites.
\item Fine-grained monitoring and aggregation of per-application metrics in a geo-distributed manner to provide end-to-end insights into application performance.
\end{itemize}
The thesis of our work is that such mechanisms are essential for meeting the quality of service guarantees for situation-awareness applications on geo-distributed infrastructures. We demonstrate by construction the efficacy and scalability of the proposed mechanisms for building dynamic compute and data orchestration policies by incorporating them in the control planes of three different platform services. Specifically, we incorporate these mechanisms into a topic-based publish-subscribe system (ePulsar), an application orchestration platform (OneEdge), and a key-value store (FogStore). We conduct extensive performance evaluation of these enhanced platform services to showcase how the new mechanisms aid in dynamically adapting the compute/data orchestration decisions to satisfy performance requirements of applications.
\end{summary}