\begin{summary}
Situation-awareness applications such as automated large-scale surveillance systems require low latency response for effective functioning and impose significant network bandwidth requirements and can benefit from the proximal nature of geo-distributed edge computing infrastructure. 
%Typical applications in this category consist of multiple compute components that process sensor streams to extract actionable insights. 
These applications often maintain state about their processing and perform real-time communication with other application instances. 
Contemporary cloud computing ecosystem offers a range of platform services to support applications with the above characteristics. Examples of such services are orchestration frameworks like Kubernetes, key-value stores such as Cassandra for storing application state and publish-subscribe systems like Pulsar and Kafka for real-time communication. These platform services perform management of compute and data while presenting intuitive and familiar abstractions so that developers can focus on core application logic. They offer a high-performance data planes that can scale to multiple nodes in a cluster. However, given that these systems are designed for datacenters, they don't provide the same performance in a much more non-uniform edge infrastructure with high client mobility. Commercial edge platform offerings such as AWS Greengrass and Azure IoT Edge, as well as academic projects such as Cloudlets offer platform services as well. However they operate on each edge site independently and are unable to scale to multiple edge locations.  
\par In this dissertation, we aim at providing platform services such as publish-subscribe systems, application orchestrators, etc. that can scale across a continuum of geo-distributed edge resources, while still being able to provide low latency compute and data access to clients by leveraging the proximity of edge locations. There are several challenges that need to be addressed in order to achieve this. First, the edge topology is highly non-uniform in terms of client-server and inter-server latencies, making topology-awareness crucial to any decision-making. Secondly, application clients are often mobile and their mobility leads to frequent changes in network connectivity and hence performance, as well as affecting the set of clients that interact with it. Finally, the unpredictable nature of client mobility, non-uniform connectivity and low resource capacity of sites creates a number of sources performance degradation, which needs to be accurately identified so that the right reconfiguration can be triggered. 

\par To this end, this dissertation proposes three mechanisms that provide vital information to the control-plane policies of geo-distributed platform services for the orchestration of compute and data entities over the infrastructure. 
\begin{itemize}
\item \textbf{Distributed Spatial Context Management} for managing the spatial context of system entities, i.e., compute and data, as well as clients. On the client-side, this mechanism manages access to the right compute and data entities corresponding to the client's current spatial context. On the control-plane-side, this mechanism manages the placement of compute and data entities.
\item \textbf{Decentralized network proximity estimation and mapping geo-location to network proximity}. This mechanism enables the control-plane of platform services to take the network topology into account for making compute/data placement decisions. For example, the topic placement policy in a publish-subscribe system can use this mechanism to ensure that end-to-end latency constraint can be met.
\item \textbf{Distributed monitoring platform for fine-grained application metrics} to enable end-to-end performance monitoring of applications. The proposed monitoring system supports custom metrics aggregation logic and alert generation logic. For instance, detecting end-to-end latency violations in a geo-distributed publish-subscribe system requires monitoring network latency between clients and broker as well as processing latency on the broker. 
\end{itemize}

The first part of this thesis describes the mechanisms in detail and the primitives that they expose to client and application components as well as the control-plane policies. It presents the architecture and implementation details of the proposed mechanisms.

The second part of this thesis describes three platform services for supporting geo-distributed situation-awareness applications that leverage the proposed mechanisms, which are enumerated as follows. (1) A topic-based publish-subscribe system that allows subscribing to topics representing spatial regions using an area-of-interest abstraction. It leverages network proximity for placement of topics across brokers and performs end-to-end monitoring to ensure publish-subscribe latency constraints are met. (2) An application orchestrator that performs latency-sensitive placement and migration of application components across edge sites. (3) A key-value store that performs latency-sensitive placement of key-value pairs along with ensuring tolerance from geographically correlated failures of edge sites. It tunes the consistency level of queries to ensure low-latency for spatially relevant items.

\end{summary}